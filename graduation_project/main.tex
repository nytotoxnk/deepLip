\documentclass[12pt]{article}

% ----------------------------------------------------------------------
% Define external packages, language, margins, fonts, new commands 
% and colors
% ----------------------------------------------------------------------
\usepackage[utf8]{inputenc} % Codification
\usepackage[english]{babel} % Writing idiom
\usepackage{multirow}
\usepackage[export]{adjustbox} % Align images
\usepackage{amsmath} % Extra commands for math mode
\usepackage{amssymb} % Mathematical symbols
\usepackage{anysize} % Personalize margins
    \marginsize{2cm}{2cm}{2cm}{2cm} % {left}{right}{above}{below}
\usepackage{appendix} % Appendices
\usepackage{cancel} % Expression cancellation
\usepackage{caption} % Captions
    \captionsetup{labelfont={bf}}
\usepackage{cite} % Citations, like [1 - 3]
\usepackage{color} % Text coloring
\usepackage{fancyvrb}
\usepackage{fancyhdr} % Head note and footnote
    \pagestyle{fancy}
    \fancyhf{}
    %\fancyhead[L]{\footnotesize Graduating } % Left of Head note
    \fancyhead[R]{\footnotesize UNYT} % Right of Head note
    \fancyfoot[L]{\footnotesize Graduating Project} % Left of Footnote
    \fancyfoot[C]{\thepage} % Center of Footnote
    \fancyfoot[R]{\footnotesize Department of Computer Science} % Right of Footnote
    \renewcommand{\footrulewidth}{0.4pt} % Footnote rule
\usepackage{float} % Utilization of [H] in figures
\usepackage{graphicx} % Figures in LaTeX
\usepackage[colorlinks = true, plainpages = true, linkcolor = istblue, urlcolor = istblue, citecolor = istblue, anchorcolor = istblue]{hyperref}
\usepackage{indentfirst} % First paragraph
\usepackage[super]{nth} % Superscripts
\usepackage{siunitx} % SI units
\usepackage{subcaption} % Subfigures
\usepackage{titlesec} % Font
    \titleformat{\section}{\Large\bfseries}{\thesection}{1em}{}
    \titleformat{\subsection}{\large\bfseries}{\thesubsection}{1em}{}
    \titleformat{\subsubsection}{\normalsize\bfseries}{\thesubsubsection}{1em}{}
    \fancyfoot[C]{\thepage}
% Random text (not needed)
\usepackage{duckuments}

% New and re-newcommands
\newcommand{\HRule}{\rule{\linewidth}{0.5mm}} % Specific rule definition
\renewcommand{\appendixpagename}{\LARGE Appendices}
\newcommand{\sectiononlytoc}[1]{%
    \par
    \refstepcounter{section}%
    % Don't print sectional unit; just add to ToC and update the Left header
    \addcontentsline{toc}{section}{\protect\numberline{\thesection}#1}%
    %\fancyhead[L]{#1}% Add chapter title in Left header
}
% Colors
\definecolor{istblue}{RGB}{3, 171, 230}
\definecolor{dkgreen}{rgb}{0,0.6,0}
\definecolor{gray}{rgb}{0.5,0.5,0.5}

%%%%%%%%%%%%%%%%%%%%%%%%%%%%%%%%%%%%%%%%%%%%%%%%%%%%%%%%%%%%%%%%%%%%%%%%
%                                 Document                             %
%%%%%%%%%%%%%%%%%%%%%%%%%%%%%%%%%%%%%%%%%%%%%%%%%%%%%%%%%%%%%%%%%%%%%%%%
\begin{document}

% ----------------------------------------------------------------------
% Cover
% ----------------------------------------------------------------------
\begin{center}
    \mbox{}\\[2.0cm]
    \LARGE Enabling Multilingual Communication: Automated Lip-synchronization Dubbing for Albanian Videos
\end{center}

\vspace{10cm}

\begin{center}
    \Large ALDO DIKU
\end{center}

\vspace{5cm}

\begin{center}
    \Large UNIVERSITY OF NEW YORK TIRANA
\end{center}
\begin{center}
    \large \Large JULY 2025
\end{center}
\thispagestyle{empty}

\setcounter{page}{0}

\newpage
% ----------------------------------------------------------------------
% APPROVAL PAGE
% ----------------------------------------------------------------------
\noindent This is to certify that I have read this project and that, in my opinion, it is fully adequate, in scope and quality, as a thesis for the degree of Bachelor of Arts in Computer Science.

\vspace{4cm}

\noindent
(Title and Name) \hspace{4cm} \\
(Project Advisor) \hspace{2cm} \underline{\hspace{2cm}} \hspace{1cm} \underline{\hspace{8cm}}

\vspace{3cm}

\noindent This is to confirm that this thesis complies with all the standards set by the Department of Computer Science of University of New York Tirana.

\vspace{2cm}

\noindent Date: \hspace{5cm} \hfill Seal/Signature: \hspace{5cm}

\setcounter{page}{1}

\newpage

% ----------------------------------------------------------------------
% PLAGARIZM CLEARANCE PAGE
% ----------------------------------------------------------------------
\noindent I hereby declare that all information in this document has been obtained and presented in accordance with academic rules and ethical conduct. I also declare that, as required by these rules and conduct, I have fully cited and referenced all material and results that are not original to this work.

\vspace{2cm}

\noindent \hspace{7.5cm} First Name, Last Name:\\
\\
\vspace{1.0cm}
\noindent \hspace{7.3cm} \text{Signature:} % NO CLUE WHY THIS IS 7.3 AND NOT 7.5 LIKE ABOVE

\setcounter{page}{2}

\newpage

% ----------------------------------------------------------------------
% ABSTRACT
% ----------------------------------------------------------------------
\sectiononlytoc{Abstract}
\begin{center}
    
    \vspace{0.5cm}
    
    ABSTRACT
    
    \vspace{1cm}
    
    \centering{ ENABLING MULTILINGUAL COMMUNICATION: AUTOMATED LIP-SYNCHRONIZATION DUBBING FOR ALBANIAN VIDEOS}
    
    \vspace{1cm}
    
    Diku, Aldo.
    
    BA. in Computer Science
    
    Thesis Advisor: Prof. Miralda Çuka
    
    July 2025, 183 pages
    
    \vspace{1cm}
\end{center}

\noindent Using available methods and tools (which i will mention when i gather all the tools and methods) to create lip-synced dubbed videos from an albanian video input and outputting a video of the same speaker in another language with the lips moving according to that languages movements. 
    
\vspace{1cm}
    
\noindent Keywords: Lip-sync, dubbing, albanian language

\newpage

% ----------------------------------------------------------------------
% Acknowledgements
% ----------------------------------------------------------------------

\sectiononlytoc{Acknowledgements}

First and foremost, I would like to express my sincere gratitude to my advisor, Professor Miralda Çuka, for their invaluable guidance, unwavering support, and insightful feedback throughout this project. Their expertise and encouragement were instrumental in shaping this work.

Finally, I am grateful for the support of my family and friends. Something something a little bit longer and better made.

\newpage


% ----------------------------------------------------------------------
% Contents
% ----------------------------------------------------------------------
\tableofcontents 

\newpage

% ----------------------------------------------------------------------
% Body
% ----------------------------------------------------------------------
\section{Introduction}

\subsection{Task description}

Option 1\\
The increasing globalisation of online content consumption presents a significant challange for video creators aiming to reach diverse audiences, thereby creating the need to produce multilingual versions of their work. This often involves an increased workload, time and other resources, a perfect example of this is a German content creator in YouTube called der8auer, who for every english video has an identical german one. This project addresses the challange of automatic translation and lip synchronization of video content into multiple languages. 
The main focus of the project will be the Albanian language as an underrepresented and low resource language. To bridge a gap in current research and practical applications of automated video translation and lip synchronization.\\
Option 2\\
Being able to make only one video and have it in multiple languages will help creators but also the target audience as well. Cutting down on the time and money it takes to make two or more videos with the same subject but in different languages. One such case is a youtube creator named der8auer from Germany, for the same topic he creates two videos, one in english and one in german. Sometimes his videos are quite long and having to do them again in another language will tire you out. Having a model that can take your video, translate it into another language and synchronize your lips to the movement of the language of your choosing would be a tool in your arsenal.\\
This project focuses more on the Albanian language as a low resource and underrepresented language, addressing a significant gap in current research and practical applications. \\

Some of the usages of this lip-sync dubbing technology are: video dubbing and translation, real-time Face-to-Face translation and multilingual communication, gaming and virtual environments (some games come in different languages but the characters are coded to move their lips only according to one language, although some companies have done work to change this depending on the language the users select, cyberpunk 2077 used JALI ai for animation lip-sync) reduced cost and labor and time in this case, entertainment and content creation, speech recognition and lip-reading.\\
----> Question: Do I need to gather datasets for the english/german/french/japanese language or are the models already capable of filling those out themselves, I am guessing not becuase from that one research paper MobLib we have a certain lip accent so the ones from the model might make the correct lip movement but it might seem not that correct for the individual.\\
Visemes are visual representations of phonemes. Phonemes are the distinct units of sound in speech, or the individual sounds that make up speech.\\
Purpose: Visemes serve to map sounds (phonemes) to corresponding mouth positions or shapes. The goal is to go from the audible (phonemes) to the visual (visemes). Visemes group similar-looking mouth shapes, which reduces the complexity required for animation. Mapping phonemes down to a smaller number of visemes gives artists fewer expressions to pose.

\subsection{Separation of work}

The first step for this project was the data creation, seeing as there is not a lot of free and publicly available free-use content I can use we had to create the data myself.\\
While there may a lot of such videos availabe in YouTube, we need to ask for permission first before having to use their videos, having to wait and then we might face rejection.\\
After we have the video data we need to get the transcription of the videos, for this to be done automatically we cannot create the videos basing them on a transcription. Several models (citations needed) were available to try for albanian transcription although with different accuracy. Automatic Speech Recognition (ASR) models are measured using Word Error Rate (WER) where a score of 0\% means perfect translation and 100\% is the worst case, working on the albanian language I focused only on those models that had support for it. Whisper by OpenAI was one such model, availabe on HuggingFace. To make such model have a better performance (need some results of the model transcribing an audio file) we need to fine-tune it. \\
Here lies another issue with working with a low resource language, not only are there low resources for video, but also for audio which we could use to fine-tune the model. There were two models made especially with albanian language in mind, albanian-asr and peshperima-v2. Because of the low ammount of resources albanian-asr was only able to get to a 46.3\% accuracy, which does not help at all. Peshperima-large-v2 was also not successful when testing. Whisper had a mulltiple sized models which required fine-tuning before being able to use it for transcription. The only trainable model size for a reasonable ammount of time was the whisper-small. The low ammount of data for the audio made the whisper-small not do much better than albanian-asr. \\
Another option was Wav2Vec and its different variations made from Facebook/Meta, this one also failed like the previous models because of lacking datasets. No model tested was able to perform in a state where they could have been used in a real world application, for that reason we needed to switch to a finished model that did not need fine-tuning.\\
Google cloud service offered speech-to-text API options and for many different languages. One of their models was offered for albanian language and had great performance, giving a confidence score about the transcription which would prove to be very usefull since you can use it as a metric to decide if you want to accept the transcription or not. \\
The next step is the preparation of the dataset collected, which was in 1-5 minute video format. From this we extracted audio from the video and tried finding moments of silence inbetween the speaking in order to cut the video into chunks to then later on feed it into the Convolutional Neural Network (CNN).
One of the reasons of why this is a hard problem is that you cannot just get the unique sounds and join them together to form a word, each unique sound changes depending on what is the letter or sound before it. Here is where the deep learning/machine learning helps as it studites large amounts of video and audio to notice these features.
\textbf{One of the challanges of creating a lip synchronization in videos is the need for the lip movements to accurately align with a specified target speech segment, especially in multilingual and unconstrained environment. Visual data falling out of sunc with updated audio and inaccurate lip movements in target videos.}\\
CNN and Generative Adversial Networks (GANs) to create the lip movements that sync up, this combined with a Discriminator Network which would try and detect the GANs fake lip movements and real ones, pushing each other to get better.

\end{document}